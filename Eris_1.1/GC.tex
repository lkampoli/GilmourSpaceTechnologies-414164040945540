% Options for packages loaded elsewhere
%\PassOptionsToPackage{unicode}{hyperref}
%\PassOptionsToPackage{hyphens}{url}
%
%\documentclass[
%]{article}
%\usepackage{amsmath,amssymb}
%\usepackage{lmodern}
%\usepackage{iftex}
%\ifPDFTeX
%  \usepackage[T1]{fontenc}
%  \usepackage[utf8]{inputenc}
%  \usepackage{textcomp} % provide euro and other symbols
%\else % if luatex or xetex
%  \usepackage{unicode-math}
%  \defaultfontfeatures{Scale=MatchLowercase}
%  \defaultfontfeatures[\rmfamily]{Ligatures=TeX,Scale=1}
%\fi
%% Use upquote if available, for straight quotes in verbatim environments
%\IfFileExists{upquote.sty}{\usepackage{upquote}}{}
%\IfFileExists{microtype.sty}{% use microtype if available
%  \usepackage[]{microtype}
%  \UseMicrotypeSet[protrusion]{basicmath} % disable protrusion for tt fonts
%}{}
%\makeatletter
%\@ifundefined{KOMAClassName}{% if non-KOMA class
%  \IfFileExists{parskip.sty}{%
%    \usepackage{parskip}
%  }{% else
%    \setlength{\parindent}{0pt}
%    \setlength{\parskip}{6pt plus 2pt minus 1pt}}
%}{% if KOMA class
%  \KOMAoptions{parskip=half}}
%\makeatother
%\usepackage{xcolor}
%\usepackage{longtable,booktabs,array}
%\usepackage{calc} % for calculating minipage widths
%% Correct order of tables after \paragraph or \subparagraph
%\usepackage{etoolbox}
%\makeatletter
%\patchcmd\longtable{\par}{\if@noskipsec\mbox{}\fi\par}{}{}
%\makeatother
%% Allow footnotes in longtable head/foot
%\IfFileExists{footnotehyper.sty}{\usepackage{footnotehyper}}{\usepackage{footnote}}
%\makesavenoteenv{longtable}
%\usepackage{graphicx}
%\makeatletter
%\def\maxwidth{\ifdim\Gin@nat@width>\linewidth\linewidth\else\Gin@nat@width\fi}
%\def\maxheight{\ifdim\Gin@nat@height>\textheight\textheight\else\Gin@nat@height\fi}
%\makeatother
%% Scale images if necessary, so that they will not overflow the page
%% margins by default, and it is still possible to overwrite the defaults
%% using explicit options in \includegraphics[width, height, ...]{}
%\setkeys{Gin}{width=\maxwidth,height=\maxheight,keepaspectratio}
%% Set default figure placement to htbp
%\makeatletter
%\def\fps@figure{htbp}
%\makeatother
%\setlength{\emergencystretch}{3em} % prevent overfull lines
%\providecommand{\tightlist}{%
%  \setlength{\itemsep}{0pt}\setlength{\parskip}{0pt}}
%\setcounter{secnumdepth}{-\maxdimen} % remove section numbering
%\ifLuaTeX
%  \usepackage{selnolig}  % disable illegal ligatures
%\fi
%\IfFileExists{bookmark.sty}{\usepackage{bookmark}}{\usepackage{hyperref}}
%\IfFileExists{xurl.sty}{\usepackage{xurl}}{} % add URL line breaks if available
%\urlstyle{same} % disable monospaced font for URLs
%\hypersetup{
%  pdftitle={Examining Spatial (Grid) Convergence},
%  hidelinks,
%  pdfcreator={LaTeX via pandoc}}
%
%\title{Examining Spatial (Grid) Convergence}
%\author{}
%\date{}
%
%\begin{document}
%\maketitle

\begin{longtable}[]{@{}
  >{\raggedright\arraybackslash}p{(\columnwidth - 0\tabcolsep) * \real{1.0000}}@{}}
\toprule()
\endhead
\begin{minipage}[t]{\linewidth}\raggedright
%\includegraphics{Examining Spatial (Grid) Convergence_files/title2.gif}
\href{https://www.grc.nasa.gov/www/wind/valid/homepage.html}{\textbf{V\&V
Home}} ~ ~ ~
\href{https://www.grc.nasa.gov/www/wind/valid/archive.html}{\textbf{Archive}}
~ ~ ~
\href{https://www.grc.nasa.gov/www/wind/valid/tutorial/tutorial.html}{\textbf{Tutorial}}

\hypertarget{examining-spatial-grid-convergence}{%
\subsection{Examining Spatial (Grid)
Convergence}\label{examining-spatial-grid-convergence}}

\hypertarget{introduction}{%
\subsubsection{Introduction}\label{introduction}}

The examination of the spatial convergence of a simulation is a
straight-forward method for determining the \emph{ordered}
\href{https://www.grc.nasa.gov/www/wind/valid/tutorial/errors.html}{discretization
error} in a CFD simulation. The method involves performing the
simulation on two or more successively finer grids. The term \emph{grid
convergence study} is equivalent to the commonly used term \emph{grid
refinement study}.

As the grid is refined (grid cells become smaller and the number of
cells in the flow domain increase) and the time step is refined
(reduced) the spatial and temporal discretization errors, respectively,
should asymptotically approaches zero, excluding computer round-off
error.

Methods for examining the spatial and temporal convergence of CFD
simulations are presented in the book by
\href{https://www.grc.nasa.gov/www/wind/valid/tutorial/bibliog.html\#Roach94}{Roache}.
They are based on use of Richardson\textquotesingle s extrapolation. A
summary of the method is presented here.

A general discussion of
\href{https://www.grc.nasa.gov/www/wind/valid/tutorial/errors.html}{errors
in CFD computations} is available for background.

We will mostly likely want to determine the error band for the
engineering quantities obtained from the finest grid solution. However,
if the CFD simulations are part of a design study that may require tens
or hundreds of simulations, we may want to use one of the coarser grids.
Thus we may also want to be able to determine the error on the coarser
grid.

\hypertarget{grid-considerations-for-a-grid-convergence-study}{%
\subsubsection{Grid Considerations for a Grid Convergence
Study}\label{grid-considerations-for-a-grid-convergence-study}}

The easiest approach for generating the series of grids is to generate a
grid with what one would consider \emph{fine} grid spacing, perhaps
reaching the upper limit of one\textquotesingle s tolerance for
generating a grid or waiting for the computation on that grid to
converge. Then coarser grids can be obtained by removing every other
grid line in each coordinate direction. This can be continued to create
additional levels of coarser grids. In generating the fine grid, one can
build in the \textbf{n} levels of coarser grids by making sure that the
number of grid points in each coordinate direction satisfies the
relation

\textbf{N = 2\textsuperscript{n} m + 1}

where \textbf{m} is an integer. For example, if two levels of coarser
grids are desired (i.e. fine, medium, and coarse grids) then the number
of grid points in each coordinate direction must equal \textbf{4 m + 1}.
The \textbf{m} may be different for each coordinate direction.

The \href{http://www.grc.nasa.gov/WWW/winddocs}{WIND} code has a
\href{http://www.grc.nasa.gov/WWW/winddocs/user/keywords/sequence.html}{grid
sequencing control} that will solve the solution on the coarser grid
without having to change the grid input file, boundary condition
settings, or the input data file. Further, the converged solution on the
coarser grid then can be used directly as the initial solution on the
finer grid. This option was originally used to speed up convergence of
solutions; however, can be used effectively for a grid convergence
study.

It is not necessary to halve the number of grid points in each
coordinate direction to obtain the coarser grid. \emph{Non-integer} grid
refinement or coarsening can be used. This may be desired since
\emph{halfing} a grid may put the solution out of the asymptotic range.
Non-integer grid refinement or coarsening will require the generation of
a new grid. It is important to maintain the same grid generation
parameters as the original grid. One approach is to select several grid
spacings as reference grid spacings. One should be the grid spacing
normal to the walls. Others may be spacings at flow boundaries, at
junctures in the geometry, or at zonal interfaces. Upon picking the
ratio as which the grid is to be refined or coarsened, this same ratio
is applied to these spacings. The number of grid points are then
adjusted according to grid quality parameters, such as grid spacing
ratio limits. The surface and volume grids are then generated using the
same methods as the original grid. The grid refinement ratio should be a
minimum of \textbf{r \textgreater= 1.1} to allow the discretization
error to be differentiated from other error sources (iterative
convergence errors, computer round-off, etc...).

\hypertarget{order-of-grid-convergence}{%
\subsubsection{Order of Grid
Convergence}\label{order-of-grid-convergence}}

The order of grid convergence involves the behavior of the solution
error defined as the difference between the discrete solution and the
exact solution,

%\includegraphics{Examining Spatial (Grid) Convergence_files/eq_error1.gif}

where \textbf{C} is a constant, \textbf{h} is some measure of grid
spacing, and \textbf{p} is the order of convergence. A "second-order"
solution would have \textbf{p = 2}.

A CFD code uses a numerical algorithm that will provide a
\emph{theoretical order of convergence}; however, the boundary
conditions, numerical models, and grid will reduce this order so that
the \emph{observed order of convergence} will likely be lower.

Neglecting higher-order terms and taking the logarithm of both sides of
the above equation results in:

%\includegraphics{Examining Spatial (Grid) Convergence_files/eq_logerror.gif}

The order of convergence \textbf{p} can be obtained from the slope of
the curve of \textbf{log(E)} versus \textbf{log(h)}. If such data points
are available, the slope can be read from the graph or the slope can be
computed from a least-squares fit of the data. The least-squares will
likely be inaccurate if there are only a few data points.

A more direct evaluation of \textbf{p} can be obtained from three
solutions using a constant grid refinement ratio \textbf{r},

%\includegraphics{Examining Spatial (Grid) Convergence_files/eq_porder1.gif}

The \emph{order of accuracy} is determined by the order of the leading
term of the \emph{truncation error} and is represented with respect to
the scale of the discretization, \textbf{h}. The \emph{local order of
accuracy} is the order for the stencil representing the discretization
of the equation at one location in the grid. The \emph{global order of
accuracy} considers the propagation and accumulation of errors outside
the stencil. This propagation causes the global order of accuracy to be,
in general, one degree less than the local order of accuracy. The order
of accuracy of the boundary conditions can be one order of accuracy
lower than the interior order of accuracy without degrading the overall
global accuracy.

\hypertarget{asymptotic-range-of-convergence}{%
\subsubsection{Asymptotic Range of
Convergence}\label{asymptotic-range-of-convergence}}

Assessing the accuracy of code and caluculations requires that the grid
is sufficiently refined such that the solution is in the asymptotic
range of convergence. The asymptotic range of convergence is obtained
when the grid spacing is such that the various grid spacings \textbf{h}
and errors \textbf{E} result in the constancy of \textbf{C},

\textbf{C = E / h\textsuperscript{p}}

Another check of the asymptotic range will be discussed in the section
on the grid convergence index.

\hypertarget{richardson-extrapolation}{%
\subsubsection{Richardson
Extrapolation}\label{richardson-extrapolation}}

Richardson extrapolation is a method for obtaining a higher-order
estimate of the continuum value (value at zero grid spacing) from a
series of lower-order discrete values.

A simulation will yield a quantity \textbf{f} that can be expressed in a
general form by the series expansion:

\textbf{f = f\textsubscript{h=0} + g\textsubscript{1} h +
g\textsubscript{2} h\textsuperscript{2} + g\textsubscript{3}
h\textsuperscript{3} + ...}

where \textbf{h} is the grid spacing and the functions
\textbf{g\textsubscript{1}}, \textbf{g\textsubscript{2}}, and
\textbf{g\textsubscript{3}} are independent of the grid spacing. The
quantity \textbf{f} is considered "second-order" if
\textbf{g\textsubscript{1} = 0.0}. The \textbf{f\textsubscript{h=0}} is
the continuum value at zero grid spacing.

If one assumes a second-order solution and has computed \textbf{f} on
two grids of spacing \textbf{h\textsubscript{1}} and
\textbf{h\textsubscript{2}} with \textbf{h\textsubscript{1}} being the
finer (smaller) spacing, then one can write two equations for the above
expansion, neglect third-order and higher terms, and solve for
\textbf{f\textsubscript{h=0}} to estimate the continuum value,

%\includegraphics{Examining Spatial (Grid) Convergence_files/eq_rich1.gif}

where the grid refinement ratio is:

\textbf{r = h\textsubscript{2} / h\textsubscript{1}}

The Richardson extrapolation can be generalized for a \textbf{p-th}
order methods and \textbf{r}-value of grid ratio (which does not have to
be an integer) as:

%\includegraphics{Examining Spatial (Grid) Convergence_files/eq_rich2.gif}

Traditionally, Richardson extrapolation has been used with grid
refinement ratios of \textbf{r = 2}. Thus, the above equation simplifies
to:

%\includegraphics{Examining Spatial (Grid) Convergence_files/eq_rich3.gif}

In theory, the above equations for the Richardson extrapolation will
provide a fourth-order estimate of \textbf{f\textsubscript{h=0}}, if
\textbf{f\textsubscript{1}} and \textbf{f\textsubscript{2}} were
computed using exactly second-order methods. Otherwise, it will be a
third-order estimate. In general, we will consider
\textbf{f\textsubscript{h=0}} to be \textbf{p+1} order accurate.
Richardson extrapolation can be applied for the solution at each grid
point, or to solution functionals, such as pressure recovery or drag.
This assumes that the solution is globally second-order in addition to
locally second-order and that the solution functionals were computed
using consistent second-order methods. Other cautions with using
Richardson extrapolation (non-conservative, amplification of round-off
error, etc...) are discussed in the book by
\href{https://www.grc.nasa.gov/www/wind/valid/tutorial/bibliog.html\#Roachebook}{Roache}.

For our purposes we will assume \textbf{f} is a solution functional
(i.e. pressure recovery). The \textbf{f\textsubscript{h=0}} is then as
an estimate of \textbf{f} in the limit as the grid spacing goes to zero.
One use of \textbf{f\textsubscript{h=0}} is to report the value as the
an improved estimate of \textbf{f\textsubscript{1}} from the CFD study;
however, one has to understand the caveats mentioned above that go along
with that value.

The other use of \textbf{f\textsubscript{h=0}} is to obtain an estimate
of the discretization error that bands \textbf{f} obtained from the CFD.
This use will now be examined.

The difference between \textbf{f\textsubscript{1}} and
\textbf{f\textsubscript{h=0}} is one error estimator; however, this
requires consideration of the caveats attached to
\textbf{f\textsubscript{h=0}}.

We will focus on using \textbf{f\textsubscript{1}} and
\textbf{f\textsubscript{2}} to obtain an error estimate. Examining the
generalized Richardson extrapolation equation above, the second term on
the right-hand side can be considered to be an an error estimator of
\textbf{f\textsubscript{1}}. The equation can be expressed as:

\textbf{A\textsubscript{1} = E\textsubscript{1} + O(
h\textsuperscript{p+1}, E\textsubscript{1}\textsuperscript{2})}

where \textbf{A\textsubscript{1}} is the actual fractional error defined
as:

\textbf{A\textsubscript{1} = ( f\textsubscript{1} - f\textsubscript{h=0}
) / f\textsubscript{h=0}}

and \textbf{E\textsubscript{1}} is the estimated fractional error for
\textbf{f\textsubscript{1}} defined as:

%\includegraphics{Examining Spatial (Grid) Convergence_files/eq_eerror1.gif}

where the relative error is defined as:

%\includegraphics{Examining Spatial (Grid) Convergence_files/eq_eps1.gif}

This quantity should not be used as an error estimator since it does not
take into account \textbf{r} or \textbf{p}. This may lead to an
underestimation or overestimation of the error. One could make this
quantity artificially small by simply using a grid refinement ratio
\textbf{r} close to 1.0.

The estimated fractional error \textbf{E\textsubscript{1}} is an
\emph{ordered} error estimator and a good approximation of the
discretization error on the fine grid if \textbf{f\textsubscript{1}} and
\textbf{f\textsubscript{2}} were obtained with good accuracy (i.e.
\textbf{E\textsubscript{1}\textless1}). The value of
\textbf{E\textsubscript{1}} may be meaningless if
\textbf{f\textsubscript{1}} and \textbf{f\textsubscript{h=0}} are zero
or very small relative to
\textbf{f\textsubscript{2}-f\textsubscript{1}}. If such is the case,
then another normalizing value should be used in place of
\textbf{f\textsubscript{1}}.

If a large number of CFD computations are to be performed (i.e for a DOE
study), one may wish to use the coarser grid with
\textbf{h\textsubscript{2}}. We will then want to estimate the error on
the coarser grid. The Richardson extrapolation can be expressed as:

%\includegraphics{Examining Spatial (Grid) Convergence_files/eq_rich4.gif}

The estimated fractional error for \textbf{f\textsubscript{2}} is
defined as:

%\includegraphics{Examining Spatial (Grid) Convergence_files/eq_eerror2.gif}

Richardson extrapolation is based on a Taylor series representation as
indicated in Eqn. \textbackslash ref\{eq:series\}. If there are shocks
and other discontinuities present, the Richardson extrapolation is
invalid in the region of the discontinuity. It is still felt that it
applies to solution functionals computed from the entire flow field.

\hypertarget{grid-convergence-index-gci}{%
\subsubsection{Grid Convergence Index
(GCI)}\label{grid-convergence-index-gci}}

\href{https://www.grc.nasa.gov/www/wind/valid/tutorial/bibliog.html\#Roach94}{Roache}
suggests a grid convergence index \textbf{GCI} to provide a consistent
manner in reporting the results of grid convergence studies and perhaps
provide an error band on the grid convergence of the solution. The
\textbf{GCI} can be computed using two levels of grid; however, three
levels are recommended in order to accurately estimate the order of
convergence and to check that the solutions are within the asymptotic
range of convergence.

A consistent numerical analysis will provide a result which approaches
the actual result as the grid resolution approaches zero. Thus, the
discretized equations will approach the solution of the actual
equations. One significant issue in numerical computations is what level
of grid resolution is appropriate. This is a function of the flow
conditions, type of analysis, geometry, and other variables. One is
often left to start with a grid resolution and then conduct a series of
grid refinements to assess the effect of grid resolution. This is known
as a grid refinement study.

One must recognize the distinction between a numerical result which
approaches an asymptotic numerical value and one which approaches the
true solution. It is hoped that as the grid is refined and resolution
improves that the computed solution will not change much and approach an
asymptotic value (i.e. the true numerical solution). There still may be
error between this asymptotic value and the true physical solution to
the equations.

\href{https://www.grc.nasa.gov/www/wind/valid/tutorial/bibliog.html\#Roach94}{Roache}
has provided a methodology for the uniform reporting of grid refinement
studies. "The basic idea is to approximately relate the results from any
grid refinement test to the expected results from a grid doubling using
a second-order method. The \textbf{GCI} is based upon a grid refinement
error estimator derived from the theory of generalized Richardson
Extrapolation. It is recommended for use whether or not Richardson
Extrapolation is actually used to improve the accuracy, and in some
cases even if the conditions for the theory do not strictly hold." The
object is to provide a measure of uncertainty of the grid convergence.

The \textbf{GCI} is a measure of the percentage the computed value is
away from the value of the asymptotic numerical value. It indicates an
error band on how far the solution is from the asymptotic value. It
indicates how much the solution would change with a further refinement
of the grid. A small value of \textbf{GCI} indicates that the
computation is within the asymptotic range.

The \textbf{GCI} on the fine grid is defined as:

%\includegraphics{Examining Spatial (Grid) Convergence_files/eq_gcifine1.gif}

where \textbf{F\textsubscript{s}} is a factor of safety. The refinement
may be spatial or in time. The factor of safety is recommended to be
\textbf{F\textsubscript{s}=3.0} for comparisons of two grids and
\textbf{F\textsubscript{s}=1.25} for comparisons over three or more
grids. The higher factor of safety is recommended for reporting purposes
and is quite conservative of the actual errors.

When a design or analysis activity will involve many CFD simulations
(i.e. DOE study), one may want to use the coarser grid
\textbf{h\textsubscript{2}}. It is then necessary to quantify the error
for the coarser grid. The \textbf{GCI} for the coarser grid is defined
as:

%\includegraphics{Examining Spatial (Grid) Convergence_files/eq_gcicoarse1.gif}

It is important that each grid level yield solutions that are in the
asymptotic range of convergence for the computed solution. This can be
checked by observing two \textbf{GCI} values as computed over three
grids,

\textbf{GCI\textsubscript{23} = r\textsuperscript{p}
GCI\textsubscript{12}}

\hypertarget{required-grid-resolution}{%
\subsubsection{Required Grid
Resolution}\label{required-grid-resolution}}

If a desired accuracy level is known and results from the grid
resolution study are available, one can then estimate the grid
resolution required to obtain the desired level of accuracy,

%\includegraphics{Examining Spatial (Grid) Convergence_files/eq_rstar.gif}

\hypertarget{independent-coordinate-refinement-and-mixed-order-methods}{%
\subsubsection{Independent Coordinate Refinement and Mixed Order
Methods}\label{independent-coordinate-refinement-and-mixed-order-methods}}

The grid refinement ratio assumes that the refinement ratio \textbf{r}
applies equally in all coordinate directions \textbf{(i,j,k)} for
steady-state solutions and also time \textbf{t} for time-dependent
solutions. If this is not the case, then the grid convergence indices
can be computed for each direction independently and then added to give
the overall grid convergence index,

%\includegraphics{Examining Spatial (Grid) Convergence_files/eq_gcisum.gif}

\hypertarget{effective-grid-refinement-ratio}{%
\subsubsection{Effective Grid Refinement
Ratio}\label{effective-grid-refinement-ratio}}

If one generates a finer or coarser grid and is unsure of the value of
grid refinement ratio to use, one can compute an effective grid
refinement ratio as:

%\includegraphics{Examining Spatial (Grid) Convergence_files/eq_reffective.gif}

where \textbf{N} is the total number of grid points used for the grid
and \textbf{D} is the dimension of the flow domain. This effective grid
refinement ratio can also be used for unstructured grids.

\hypertarget{example-grid-convergence-study}{%
\subsubsection{Example Grid Convergence
Study}\label{example-grid-convergence-study}}

The following example demonstrates the application of the above
procedures in conducting a grid convergence study. The objective of the
CFD analysis was to determine the pressure recovery for an inlet. The
flow field is computed on three grids, each with twice the number of
grid points in the \textbf{i} and \textbf{j} coordinate directions as
the previous grid. The number of grid points in the \textbf{k} direction
remains the same. Since the flow is axisymmetric in the \textbf{k}
direction, we consider the finer grid to be double the next coarser
grid. The table below indicates the grid information and the resulting
pressure recovery computed from the solutions. Each solution was
properly converged with respect to iterations. The column indicated by
"spacing" is the spacing normalized by the spacing of the finest grid.

%\begin{longtable}[]{@{}ccc@{}}
%\toprule()
%\endhead
%Grid & Normalized Grid Spacing & Pressure Recovery, Pr \\
%1 & 1 & 0.97050 \\
%2 & 2 & 0.96854 \\
%3 & 4 & 0.96178 \\
%\bottomrule()
%\end{longtable}

The figure below shows the plot of pressure recoveries with varying grid
spacings. As the grid spacing reduces, the pressure recoveries approach
an asymptotic zero-grid spacing value.

We determine the order of convergence observed from these results,

\textbf{p = ln{[} ( 0.96178 - 0.96854 ) / ( 0.96854 - 0.97050 ) {]} /
ln(2) = 1.786170}

The theoretical order of convergence is \textbf{p=2.0}. The difference
is most likely due to grid stretching, grid quality, non-linearities in
the solution, presence of shocks, turbulence modeling, and perhaps other
factors.

We now can apply Richardson extrapolation using the two finest grids to
obtain an estimate of the value of the pressure recovery at zero grid
spacing,

\textbf{Pr\textsubscript{h=0} = 0.97050 + ( 0.97050 - 0.96854 ) / (
2\textsuperscript{1.786170} - 1 )\\
= 0.97050 + 0.00080 = 0.97130}

This value is also plotted on the figure below.

The grid convergence index for the fine grid solution can now be
computed. A factor of safety of \textbf{F\textsubscript{S}=1.25} is used
since three grids were used to estimate \textbf{p}. The \textbf{GCI} for
grids 1 and 2 is:

\textbf{GCI\textsubscript{12} = 1.25 \textbar{} ( 0.97050 - 0.96854 ) /
0.97050 \textbar{} / ( 2\textsuperscript{1.786170} - 1 ) 100\% =
0.103083\%}

The \textbf{GCI} for grids 2 and 3 is:

\textbf{GCI\textsubscript{23} = 1.25 \textbar{} ( 0.96854 - 0.96178 ) /
0.96854 \textbar{} / ( 2\textsuperscript{1.786170} - 1 ) 100\% =
0.356249\%}

We can now check that the solutions were in the asymptotic range of
convergence,

\textbf{0.356249 / ( 2\textsuperscript{1.786170} 0.103083 ) = 1.002019}

which is approximately one and indicates that the solutions are well
within the asymptotic range of convergence.

Based on this study we could say that the pressure recovery for the
supersonic ramp is estimated to be \textbf{Pr = 0.97130} with an error
band of \textbf{0.103\%} or \textbf{0.001}.

%\includegraphics{Examining Spatial (Grid) Convergence_files/pr.gif}

\hypertarget{verify-a-fortran-program-to-perform-calculations-associated-with-a-grid-convergence-study}{%
\subsubsection{VERIFY: A Fortran program to Perform Calculations
Associated with a Grid Convergence
Study}\label{verify-a-fortran-program-to-perform-calculations-associated-with-a-grid-convergence-study}}

The Fortran 90 program
\href{https://www.grc.nasa.gov/www/wind/valid/tutorial/verify.f90}{verify.f90}
was written to carry out the calculations associated with a grid
convergence study involving 3 or more grids. The program is compiled on
a unix system through the commands:

\begin{quote}
\textbf{f90 verify.f90 -o verify}
\end{quote}

It reads in an ASCII file
(\href{https://www.grc.nasa.gov/www/wind/valid/tutorial/prD.do}{prD.do})
through the standard input unit (5) that contains a list of pairs of
grid size and value of the observed quantity \textbf{f}.

\begin{quote}
\textbf{verify \textless{}
\href{https://www.grc.nasa.gov/www/wind/valid/tutorial/prD.do}{prD.do}
\textgreater{}
\href{https://www.grc.nasa.gov/www/wind/valid/tutorial/prD.out}{prD.out}}
\end{quote}

It assumes the values from the finest grid are listed first. The output
is then written to the standard output unit (6)
\href{https://www.grc.nasa.gov/www/wind/valid/tutorial/prD.out}{prD.out}.
The output from the of \{\textbackslash tt verify\} for the results of
Appendix A are:

\begin{quote}
\begin{verbatim}
 --- VERIFY: Performs verification calculations ---
  
 Number of data sets read =  3
  
      Grid Size     Quantity
  
      1.000000      0.970500
      2.000000      0.968540
      4.000000      0.961780
  
 Order of convergence using first three finest grid 
 and assuming constant grid refinement (Eqn. 5.10.6.1) 
 Order of Convergence, p =  1.78618479
  
 Richardson Extrapolation: Use above order of convergence
 and first and second finest grids (Eqn. 5.4.1) 
 Estimate to zero grid value, f_exact =  0.971300304
  
 Grid Convergence Index on fine grids. Uses p from above.
 Factor of Safety =  1.25
  
   Grid     Refinement            
   Step      Ratio, r       GCI(%)
   1  2      2.000000      0.103080
   2  3      2.000000      0.356244
  
 Checking for asymptotic range using Eqn. 5.10.5.2.
 A ratio of 1.0 indicates asymptotic range.
  
  Grid Range    Ratio
  12 23      0.997980
  
 --- End of VERIFY ---
\end{verbatim}
\end{quote}

\hypertarget{examples-of-grid-converence-studies-in-the-archive}{%
\subsubsection{Examples of Grid Converence Studies in the
Archive}\label{examples-of-grid-converence-studies-in-the-archive}}

A grid convergence study is performed in the
\href{https://www.grc.nasa.gov/www/wind/valid/tutorial/wedge/wedge.html}{Supersonic
Wedge} case.

\hypertarget{examples-of-grid-converence-studies-in-literature}{%
\subsubsection{Examples of Grid Converence Studies in
Literature}\label{examples-of-grid-converence-studies-in-literature}}

Other examples of grid convergence studies that use the procedures
outlined above can be found in
\href{https://www.grc.nasa.gov/www/wind/valid/tutorial/bibliog.html\#Roachebook}{the
book by Roache} and the paper by
\href{https://www.grc.nasa.gov/www/wind/valid/tutorial/bibliog.html\#Steffen95}{Steffen
et al.}.

\hypertarget{nparc-alliance-policy-with-respect-to-grid-converence-studies}{%
\subsubsection{NPARC Alliance Policy with Respect to Grid Converence
Studies}\label{nparc-alliance-policy-with-respect-to-grid-converence-studies}}

For the WIND verification and validation effort, it is suggested that
the above procedures be used when conducting and reporting results from
a grid convergence study.

\begin{center}\rule{0.5\linewidth}{0.5pt}\end{center}

{Last Updated:} Wednesday, 10-Feb-2021 09:38:58 EST\\
\strut
\end{minipage} \\
\bottomrule()
\end{longtable}

\hypertarget{footer}{}
{Responsible NASA Official/Curator:}
\href{mailto:john.w.slater@nasa.gov}{John W. Slater}

{{Web Policies: }
\href{http://www.grc.nasa.gov/Doc/grcwebpolicies.html}{Web Privacy
Policy and Important Notices}}\\
{Adobe Reader Download: }
\href{http://www.adobe.com/products/acrobat/readstep2.html}{Adobe
Reader}

%\end{document}